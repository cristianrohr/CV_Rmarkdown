%%%%%%%%%%%%%%%%%
% This is an sample CV template created using altacv.cls
% (v1.1.5, 1 December 2018) written by LianTze Lim (liantze@gmail.com). Now compiles with pdfLaTeX, XeLaTeX and LuaLaTeX.
%
%% It may be distributed and/or modified under the
%% conditions of the LaTeX Project Public License, either version 1.3
%% of this license or (at your option) any later version.
%% The latest version of this license is in
%%    http://www.latex-project.org/lppl.txt
%% and version 1.3 or later is part of all distributions of LaTeX
%% version 2003/12/01 or later.
%%%%%%%%%%%%%%%%

%% If you need to pass whatever options to xcolor
\PassOptionsToPackage{dvipsnames}{xcolor}

%% If you are using \orcid or academicons
%% icons, make sure you have the academicons
%% option here, and compile with XeLaTeX
%% or LuaLaTeX.
% \documentclass[10pt,a4paper,academicons]{altacv}

%% Use the "normalphoto" option if you want a normal photo instead of cropped to a circle
% \documentclass[10pt,a4paper,normalphoto]{altacv}

\documentclass[10pt,a4paper,ragged2e]{altacv}

%% AltaCV uses the fontawesome and academicon fonts
%% and packages.
%% See texdoc.net/pkg/fontawecome and http://texdoc.net/pkg/academicons for full list of symbols. You MUST compile with XeLaTeX or LuaLaTeX if you want to use academicons.

% Change the page layout if you need to
\geometry{left=1cm,right=9cm,marginparwidth=6.8cm,marginparsep=1.2cm,top=1.25cm,bottom=1.25cm}

% Change the font if you want to, depending on whether
% you're using pdflatex or xelatex/lualatex
\ifxetexorluatex
  % If using xelatex or lualatex:
  \setmainfont{Lato}
\else
  % If using pdflatex:
  \usepackage[utf8]{inputenc}
  \usepackage[T1]{fontenc}
  \usepackage[default]{lato}
\fi

% Change the colours if you want to
\definecolor{airforceblue}{rgb}{0.36, 0.54, 0.66}
\definecolor{blue(ncs)}{rgb}{0.0, 0.53, 0.74}
\definecolor{CRColor}{HTML}{6699ff}

\definecolor{reglas}{HTML}{5588a3}
\definecolor{head1}{HTML}{145374}

\definecolor{Mulberry}{HTML}{72243D}
\definecolor{SlateGrey}{HTML}{2E2E2E}
\definecolor{LightGrey}{HTML}{666666}

\colorlet{heading}{head1}
\colorlet{accent}{reglas}

%\colorlet{heading}{Sepia}
%\colorlet{accent}{Mulberry}
\colorlet{emphasis}{SlateGrey}
\colorlet{body}{LightGrey}

% Change the bullets for itemize and rating marker
% for \cvskill if you want to
\renewcommand{\itemmarker}{{\small\textbullet}}
\renewcommand{\ratingmarker}{\faCircle}

%% sample.bib contains your publications
%\usepackage[citestyle=authoryear,bibstyle=authoryear,
%             maxbibnames=10,giveninits=true,natbib=true,doi=false,url=false,
%             isbn=false,dashed=false,uniquelist=false,uniquename=false,backend=biber]{biblatex}
\usepackage[backend=biber]{biblatex}
\addbibresource{citasCR.bib}

\begin{document}
\name{Cristian Rohr}
\tagline{Msc. Ciencia de Datos | Lic. Bioinformática}
\photo{2.8cm}{CR_foto}
\personalinfo{%
  % Not all of these are required!
  % You can add your own with \printinfo{symbol}{detail}
  \email{cristianrohr768@gmail.com - cristianrohr@correo.ugr.es}
  \phone{+54 9 3434164989}
  \mailaddress{Rodriguez 1235 1A}
  \location{Rosario, Argentina}
  \homepage{cristianrohr.github.io}
%  \twitter{@twitterhandle}
  \linkedin{linkedin.com/in/cristianrohrbio}
  \github{github.com/cristianrohr}
  %% You MUST add the academicons option to \documentclass, then compile with LuaLaTeX or XeLaTeX, if you want to use \orcid or other academicons commands.
  % \orcid{orcid.org/0000-0000-0000-0000}
}

%% Make the header extend all the way to the right, if you want.
\begin{fullwidth}
\makecvheader
\end{fullwidth}

%% Depending on your tastes, you may want to make fonts of itemize environments slightly smaller
% \AtBeginEnvironment{itemize}{\small}

%% Provide the file name containing the sidebar contents as an optional parameter to \cvsection.
%% You can always just use \marginpar{...} if you do
%% not need to align the top of the contents to any
%% \cvsection title in the "main" bar.

\cvsection[sample-p1sidebar]{Resumen}
Más de 8 años de experiencia en el análisis de datos, tanto en la industria donde formé parte de equipos interdisciplinarios en el área de agrobiotecnología y medicina traslacional, donde lideré el desarrollo de dos productos disponibles comercialmente,
como así también en el ámbito académico donde tengo un buen registro de publicaciones, presentaciones en congreso
y experiencia docente. 

%Mis intereses se centran en dos áreas, la aplicación de técnicas de obtención, procesamiento y modelado de datos con aplicación en problemas prácticos y en la medicina genómica.
Mis intereses se centran principalmente en el machine learning y big data, aplicados a la medicina genómica y agrobiotecnología.

\medskip

\cvsection{Experiencia}

\cvevent{Especialista Ciencia de Datos - Líder Bioinformática}{Héritas}{Octubre 2018 -- Actualidad}{Rosario, Argentina}
%\begin{itemize}
%\item 
En el tiempo que desarrolle el máster serví como consultor a distancia.
%\end{itemize}

\divider

\cvevent{Bioinformático / Data Scientist }{Héritas/INDEAR}{Abril 2016 -- Sept. 2018}{Rosario, Argentina}
\begin{itemize}
\item Desarrollo, implementación y puesta en producción de los algoritmos y modelos predictivos del test prenatal no invasivo
Héritas VISION, y del test de microbioma intestinal Héritas MicroXplora.
\item Desarrollo de herramientas y paquetes en Shiny/R para automatizar tareas y asistir a los análistas genéticos, en diversos
productos como el test de biopsías líquidas Héritas OncoSens, Héritas Focus, Héritas CLEAR
\item Diseño, desarrollo, implementación y mantenimiento de pipelines de análisis genómicos en entornos de computo de
alto rendimiento diversas aplicaciones de I+D internas de la empresa: metagenómica 16S, RNAseq, ensamblado de genomas bacterianos, genotipado por secuenciación.
\item Análisis de secuencias de inserción en eventos transgénicos para su desregulación en mercados internacionales.
\item Asesoramiento a los usuarios del área de genómica del Instituto de Agrobiotecnología de Rosario.
\end{itemize}

\cvsection{Experiencia Académica}

\cvevent{Asistente Investigación}{Laboratorio de Genómica Médica y Evolución, FCEyN - UBA}{Abril 2013 -- Abril 2016}{Buenos Aires, Argentina}
%\begin{itemize}
%\item Colaboración en proyectos de investigación e I+D con diversas instituciones del ámbito público y privado.
%\item Desarrollo e implementación de bases de datos y servidores web para el análisis/visualización de datos obtenidos con
%paneles de secuenciación de tecnologías Ion Torrent de Life Technologies.
%\item Desarrollo de paneles personalizados para el screening de enfermedades humanas.
%\end{itemize}

\divider


\cvevent{Asistente de Investigación}{Laboratorio de Agrobiotecnología - FCEyN, UBA}{Septiembre 2011 -- Marzo 2013}{Buenos Aires, Argentina}
%\begin{itemize}
%\item Participe en diversos proyectos del laboratorio como ser RNAseq y creación de bases de datos.
%\item Implementación de soluciones bioinformáticas para actividades rutinarias desarrolladas en el laboratorio.
%\end{itemize}

\clearpage
\cvsection[page2sidebar]{Docencia}

\cvevent{Jefe de Trabajos Prácticos}{Universidad de Buenos Aires}{2015}{Buenos Aires, Argentina}
%\begin{itemize}
%\item 
Curso de posgrado: “Genómica de poblaciones y enfermedades”. Departamento de Ecología, Genética y Evolución, FCEyN
%\end{itemize}

\divider

\cvevent{Jefe de Trabajos Prácticos}{Universidad de Buenos Aires}{2014}{Buenos Aires, Argentina}
%\begin{itemize}
%\item 
Curso de posgrado: “Genómica humana: variación, adaptación y poblaciones”. Departamento de Ecología, Genética y Evolución, FCEyN
%\end{itemize}

\divider

\cvevent{Docente Invitado}{RSG Argentina}{2012}{}
%\begin{itemize}
%\item 
Curso "Perl y Bioperl aplicado en bioinformática"
%\end{itemize}

\cvsection{Presentaciones en congresos}

\cvevent{Héritas VISION: The challenge to develop the first Non Invasive Prenatal Testing (NIPT) Platform in Argentina}{8vo Congreso Argentino de Bioinformática y Biología Computacional (8CA2BC)}{26-29 Noviembre de 2017}{Posadas, Argentina}

\cvevent{Using Oxford Nanopore MinION technology to deeply explore metagenome functions in the Argentine Human microbiome dataset characterised by Illumina 16s metagenomics pipeline}{Fourth International Society for Computational Biology Latin America Bioinformatics Conference (ISCB-LA) and the 7th Argentinian Congress of Bioinformatics and Computational Biology (CAB2C)}{21-23 Noviembre de 2016}{Buenos Aires, Argentina}

\cvevent{INSECT: Una herramienta para descomponer redes complejas de regulación génica}{1er Encuentro de estudiantes de bioinformática y biología computacional}{25-26 Noviembre de 2014}{Buenos Aires, Argentina}


\cvevent{A Web Server for the Standardization of Molecular Diagnosis of Huntington’s Disease in Latin America}{IV Argentinean Conference on Computational Biology and Bioinformatics \& the IV Conference of the Iberoamerican Society for Bioinformatics}{29-31 Octubre de 2013}{Rosario, Argentina}


\cvevent{Bioinformatics annotation of Pycnoporus sanguineus BAFC 2126 transcriptome}{IV Argentinean Conference on Computational Biology and Bioinformatics \& the IV Conference of the Iberoamerican Society for Bioinformatics}{29-31 Octubre de 2013}{Rosario, Argentina}





%\clearpage
%\cvsection{Publicaciones}

%\divider

%\printbibliography[heading=pubtype,title={\printinfo{\faGroup}{Conference Proceedings}},type=inproceedings]

%% If the NEXT page doesn't start with a \cvsection but you'd
%% still like to add a sidebar, then use this command on THIS
%% page to add it. The optional argument lets you pull up the
%% sidebar a bit so that it looks aligned with the top of the
%% main column.
% \addnextpagesidebar[-1ex]{page3sidebar}

\clearpage
\cvsection[page3sidebar]{Publicaciones}


\nocite{*}

%\printbibliography[heading=pubtype,title={\printinfo{\faBook}{Books}},type=book]

%\divider

\printbibliography[heading=pubtype,title={\printinfo{\faFileTextO}{Revistas internacionales}},type=article]

%\divider

%\printbibliography[heading=pubtype,title={\printinfo{\faGroup}{Conference Proceedings}},type=inproceedings]


\end{document}
